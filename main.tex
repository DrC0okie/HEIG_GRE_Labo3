\documentclass[11pt, a4paper]{article}
  \usepackage[T1]{fontenc}
  \usepackage[utf8]{inputenc}
  \usepackage[french]{babel}
  \usepackage{lmodern}
  \usepackage[usegeometry]{typearea}
  \usepackage{geometry}
  \usepackage{esvect}
  \usepackage{amsmath}
  \usepackage{amsthm}
  \usepackage{amssymb}
  \usepackage{amsfonts}
  \usepackage{multicol}
  \usepackage{color}
  \usepackage{graphicx}
  \usepackage{overpic}
  \usepackage{enumitem}
  \usepackage{colortbl}
  \usepackage{wasysym}
  \usepackage{tabularx}
  \usepackage{csquotes}
  \usepackage{tikz}
  \usepackage{float}
  \usepackage{systeme}
  \usepackage{titlesec}
  \usepackage{pgf}
  \usepackage{bm}
  \usepackage{tikz}
  \usepackage{pgfplots}
  \usepackage{mathrsfs}
  \usepackage{framed}
  \usepackage{remreset}
  \usepackage{array}
  \usepackage{calc}
  \usepackage{ifthen}
  \usepackage{polynom}
  \usepackage{xhfill}
  \usepackage{fancyhdr}
  \usepackage{xfrac}
  \usepackage{chemarr}
  \usepackage{mleftright}
  \usepackage{cancel}
  \usepackage{romanbar}
  \usepackage{booktabs}
  \usepackage{siunitx}
  \usepackage{gensymb}
  \usepackage{wrapfig}
  \usepackage{etoolbox}
  \usepackage{soul}
  \usepackage{listings}
  \usepackage{microtype}
  \usepackage{pdfpages}
  \usepackage{minted}
  \usepackage[framemethod=tikz]{mdframed}
  \usepackage[nodayofweek]{datetime}
  \usepackage[bookmarks=false, colorlinks]{hyperref}
  \usetikzlibrary{arrows}
  \usetikzlibrary{babel}
  \setlength{\parindent}{0pt}
  \pgfplotsset{compat=1.14}
  \polyset{%
    style=C,
    delims={\big(}{\big)}
  }

  \newcommand{\hSchool}{HEIG-VD}
  \newcommand{\hClass}{GRE}
  \newcommand{\hFile}{Problème d'assemblée de délégués}
  \newcommand{\hFileType}{}
  \newcommand{\hFileShort}{Laboratoire 3}
  \newcommand{\hAuthors}{Jarod Streckeisen, Timothée Van Hove}
  \newcommand{\hAuthorsFancy}{Jarod \textsc{Streckeisen}, Timothée \textsc{Van Hove}}
  \newcommand{\hPdfTitle}{\hSchool - \hClass - \hFile :: \hFileType}

  \geometry{
    a4paper,
    left=15mm,
    right=15mm,
    marginpar=20mm,
    top=18mm,
    bottom=22mm,
    %showframe
  }
  
  \hypersetup{
    pdftitle={\hPdfTitle},
    pdfauthor={\hAuthors},
    pdfsubject={\hClass},
    pdfkeywords={},
    pdfcreator={LaTeX},
    pdfproducer={pdfLaTeX},
    linkcolor=black,
    citecolor=black,
    filecolor=black,
    urlcolor=black
  }


  \renewcommand{\baselinestretch}{1.05}
  % \setlength{\columnseprule}{1pt}
  \setlength{\parindent}{0pt}
  \setlength{\parskip}{3pt plus 0.5ex}
  % \setcounter{secnumdepth}{0}
  \everymath{\displaystyle}
  
  \newcommand{\g}[1]{\og #1 \fg{}}
  \newcommand{\HRule}{\rule{.9\linewidth}{.6pt}}
  \newcommand\tab[1][1cm]{\hspace*{#1}}

  \title{\vspace{-2.25cm}\hSchool{} --- \hClass\\\textbf{\hFile{}  \hFileType}\vspace{-0.45cm}}
  \date{\vspace{-0.35cm}\today\\\vspace{-0.5cm}}
  \author{\hAuthorsFancy}
  
  \pagestyle{fancy}
  \fancyhf{}
  \rhead{\hAuthors}
  \lhead{\hClass{} -- \hFileShort}
  \lfoot{\twodigit{\the\month}.\the\year}
  \rfoot{\thepage}
  
  \definecolor{wrwrwr}{rgb}{0.38,0.38,0.38}
  \definecolor{ffxfqq}{rgb}{1.,0.50,0.}
  \definecolor{e8000d}{rgb}{0.91,0.,0.13}
  \definecolor{aqaqaq}{rgb}{0.63,0.63,0.63}
  \definecolor{000000}{rgb}{0.,0.,0.}
  \definecolor{ffffff}{rgb}{1.,1.,1.}
  
\definecolor{pblue}{rgb}{0.13,0.13,1}
\definecolor{pgreen}{rgb}{0,0.5,0}
\definecolor{pred}{rgb}{0.9,0,0}
\definecolor{pgrey}{rgb}{0.46,0.45,0.48}
  
  \setitemize{label=--}

  %\preto{\section}{\clearpageafterfirst}
  %\preto{\subsection}{\filbreak}
  % \newcommand{\clearpageafterfirst}{%
  %   \gdef\clearpageafterfirst{\clearpage}%
  % }

  \setcounter{secnumdepth}{4}
  \titleformat{\section}
    {\vspace{-0.15cm}\normalfont\large\bfseries}{\thesection}{1em}{}[\vspace{-0.15cm}]
  \titleformat{\subsection}
    {\vspace{-0.2cm}\normalfont\small\bfseries}{\thesubsection}{1em}{}[\vspace{-0.05cm}]
  \titleformat{\subsubsection}
    {\vspace{-0.15cm}\normalfont\small\bfseries}{\thesubsubsection}{1em}{}[\vspace{-0.05cm}]
  \titleformat{\paragraph}
    {\vspace{-0.15cm}\normalfont\small\bfseries}{\theparagraph}{1em}{}[\vspace{-0.15cm}]
    
    \setlist{nosep}

\begin{document}
\maketitle

\section{Description du problème}
L’assemblée des délégués, organe central de l’association cantonale vaudoise de gymnastique, est composée, entre autres, d’un membre de chacune des $p$ sociétés de gymnastique du canton. Parmi tous les membres de ces sociétés, $n$ se sont portés volontaires pour siéger à l’assemblée des délégués. Pour chaque candidat, on connaît son district de résidence et les sociétés dont il est membre (certaines personnes sont membres de plusieurs clubs de gymnastique). 

L'objectif est de créer, si possible, une assemblée formée de $p$ membres, un par société de gymnastique, et assurant pour chacun des 10 districts du canton un nombre de délégués compris entre un minimum $l_j$ et un maximum $u_j$ (bornes incluses), $j = 1, \ldots, 10$.

Nous supposons que $n \geq p > 1$ et que la somme $\sum_{j=1}^{10} l_j \leq p \leq \sum_{j=1}^{10} u_j$.

\section{Modélisation du problème}
Nous avons décidé de modéliser ce problème sous la forme d’un problème de flots dans un réseau.

\subsection{Construction du réseau}

Construire le réseau $R = (V, E, u)$  comme suit:

\begin{itemize}
    \item Un sommet $s$, représentant la source.
    \item Un sommet $t$, représentant le puits.
    \item Les sommets $S_i$ représentant les sociétés, pour $i = 1, \ldots, p$.
    \item Les sommets $C_i$ représentant les candidats, pour $i = 1, \ldots, n$.
    \item Les sommets $D_j$ représentant les districts, pour $j = 1, \ldots, 10$.
    \item Deux sommets fictifs $T'$ et $T''$ pour gérer les contraintes de représentation minimale et maximale par district.
    \item De la source $s$ à chaque $S_i$, une arête de capacité $1$.
    \item De chaque $S_i$ à chaque $C_i$, une arête de capacité $1$ si le candidat $C_i$ est membre de la société $S_i$.
    \item De chaque $C_i$ à son district $D_j$, une arête de capacité $1$.
    \item De chaque $D_j$ à $T'$, une arête de capacité $l_j$ (le minimum de représentants pour le district $j$).
    \item De chaque $D_j$ à $T''$, une arête de capacité $u_j - l_j$ (la différence entre le maximum et le minimum de représentants pour le district $j$).
    \item De $T'$ à $t$, une arête de capacité $\sum_{j=1}^{10} l_j$.
    \item De $T''$ à $t$, une arête de capacité $p - \sum_{j=1}^{10} l_j$.
\end{itemize}
\pagebreak
\subsection {Illustration d'une construction d'un tel réseau}

\begin{figure}[H]
    \centering
    \includegraphics[width=1\linewidth]{GRE-diagram.png}
    \caption{Exemple de Réseau}
    \label{fig:exemple-reseau}
\end{figure}

\section{Résolution du problème}
Pour résoudre ce problème, nous utilisons un algorithme pour le calcul d’un flot compatible de valeur maximale (comme l'algorithme de Ford-Fulkerson) pour trouver le flot maximum de $s$ à $t$ dans $R$.

\subsection{Interprétation des résultats}
\begin{itemize}
    \item Si le flot maximum trouvé est égal à $p$, alors il est possible de sélectionner $p$ candidats respectant toutes les contraintes de représentation.
    \item Si le flot maximum est inférieur à $p$, alors il n'est pas possible de former une assemblée respectant toutes les contraintes.
\end{itemize}

Dans le cas où le flot maximum est égal à $p$, nous analysons les arêtes saturées pour déterminer les candidats sélectionnés pour former l'assemblée.


\end{document}
